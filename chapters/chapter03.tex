\section{Data}

\begin{table}[h]
\caption{The List of Languages}
\label{tab:language_list}
\resizebox{\textwidth}{!}{\begin{tabular}{|l|l|l|l|l|}
\hline
\textbf{code} & \textbf{Name}                                                          & \textbf{Genus}                                                                   & \textbf{Family}               & \textbf{Reference} \\ \hline
fij  & Fijian                                                        & Oceanic                                                                 & Austronesian         &    \citet{dixon1988grammar}       \\ \hline
tha  & Thai                                                          & Kam-Tai                                                                 & Tai-Kadai            &  \citet{tingsabadhThai1993}         \\ \hline
hau  & Hausa                                                         & West Chadic                                                             & Afro-Asiatic         &     \citet{newmanHausaLanguageEncyclopedic2000}      \\ \hline
lav  & Lavukaleve                                                    & Lavukaleve                                                              & Solomons East Papuan &        \citet{terrill2011grammar}   \\ \hline
qim  & \begin{tabular}[c]{@{}l@{}}Quechua \\ (Imbabura)\end{tabular} & Quechuan                                                                & Quechuan             &     \citet{coleImbaburaQuechua1985}      \\ \hline
may  & Maybrat                                                       & Maybrat                                                                 & Maybrat              &     \citet{dolGrammarMaybratLanguage2007}      \\ \hline
imo  & Imonda                                                        & Border                                                                  & Border               &      \citet{seilerImondaPapuanLanguage1985}     \\ \hline
kio  & Kiowa                                                         & Kiowa-Tanoan                                                            & Kiowa-Tanoan         &      \citet{watkinsGrammarKiowa1984}     \\ \hline
mnd  & Mandarin                                                      & Chinese                                                                 & Sino-Tibetan         &     \citet{liMandarinChineseFunctional2009}      \\ \hline
spa  & Spanish                                                       & Romance                                                                 & Indo-European        &      \citet{geeslinCambridgeHandbookSpanish2018}     \\ \hline
war  & Wari'                                                         & Chapacura-Wanham                                                        & Chapacura-Wanham     &      \citet{everettWari2002}     \\ \hline
kor  & Korean                                                        & Korean                                                                  & Korean               &     \citet{shinSoundsKorean2012}      \\ \hline
vie  & Vietnamese                                                    & Viet-Muong                                                              & Austro-Asiatic       &     \citet{kirbyVietnameseHanoiVietnamese2011}      \\ \hline
chk  & Chukchi                                                       & \begin{tabular}[c]{@{}l@{}}Northern \\ Chukotko-Kamchatkan\end{tabular} & Chukotko-Kamchatkan  &      \citet{dunnGrammarChukchi1999}     \\ \hline
wra  & Warao                                                         & Warao                                                                   & Warao                &      \citet{romero-figeroaReferenceGrammarWarao2003}     \\ \hline
knd  & Kannada                                                       & Dravidian                                                               & Dravidian            &      \citet{sridharKannada1990}     \\ \hline
sup  & Supyire                                                       & Senufo                                                                  & Niger-Congo          &     \citet{carlsonGrammarSupyire1994}      \\ \hline
tiw  & Tiwi                                                          & Tiwian                                                                  & Tiwian               &     \citet{osborneTiwiLanguageGrammar1974}      \\ \hline
lan  & Lango                                                         & Western Nilotic                                                         & Eastern Sudanic      &     \citet{noonanGrammarLango1992}      \\ \hline
kew  & Kewa                                                          & Enga\_Kewa-Huli                                                         & Trans-New Guinea     &     \citet{franklinGrammarKewaNew1971} \\ \hline
\end{tabular}}
\end{table}

The 20 languages used in the survey are presented in Table \ref{tab:language_list} along with their genealogical data and the reference used. 
A total of 411 tokens of 95 different segments was recorded, the 20 most common segments and their restriction percentages are presented in Table \ref{tab:segment_list}. The full list of all segments is provided in Appendix B. 

\begin{table}[]
\centering
\caption{List of most common segments}
\label{tab:segment_list}
\begin{tabular}{|l|l|l|l|l|}
\hline
Segment & Count & Initial & Medial & Final \\ \hline
m        & 20    & 1.00    & 1.00   & 0.94  \\ \hline
n        & 20    & 1.00    & 1.00   & 0.94  \\ \hline
t        & 20    & 1.00    & 0.90   & 0.56  \\ \hline
k        & 19    & 1.00    & 0.95   & 0.47  \\ \hline
p        & 18    & 0.94    & 0.89   & 0.43  \\ \hline
s        & 18    & 0.94    & 1.00   & 0.57  \\ \hline
l        & 16    & 0.94    & 0.94   & 0.62  \\ \hline
j        & 15    & 0.93    & 0.93   & 0.55  \\ \hline
d        & 13    & 0.92    & 0.85   & 0.20  \\ \hline
w        & 13    & 1.00    & 1.00   & 0.56  \\ \hline
b        & 12    & 0.92    & 0.92   & 0.30  \\ \hline
ŋ        & 12    & 0.67    & 0.67   & 1.00  \\ \hline
g        & 11    & 0.82    & 0.91   & 0.22  \\ \hline
f        & 10    & 1.00    & 0.90   & 0.63  \\ \hline
h        & 10    & 1.00    & 0.90   & 0.11  \\ \hline
r        & 10    & 0.90    & 1.00   & 0.57  \\ \hline
ɾ        & 8     & 0.63    & 1.00   & 0.71  \\ \hline
tʰ       & 8     & 0.88    & 0.88   & 0.25  \\ \hline
tʃ       & 7     & 1.00    & 1.00   & 0.00  \\ \hline
ʔ        & 7     & 0.86    & 1.00   & 0.40  \\ \hline
\end{tabular}
\end{table}

\section{Overview}
The list of 20 most frequent segments in this study is similar to the top 20 segments list reported in \citet[45]{Gordon_2016}, which comes from a survey of 317 languages. The only two different segments are /ʃ/ and /ɲ/ in his list compared to /r/ and /tʰ/ in my list. There are even some resemblance in terms of the frequency of each segmentsː /n t m k/ are the four most common segments in both lists. Therefore, dispite the small scale of this study the list of top 20 segments can still be considered representative of world languages to some degree. 

\par
For most segments in the survey, we can see that the distributional discrepancies between different positions are apparent. 
The most symmetric segments are /n/ and /m/, which have equal occurrences in initial and medial positions, with slightly fewer instances in the final position. 
The distribution of velar nasal /ŋ/ is consistent with the finding of \citet{wals-9}, namely that all languages that have /ŋ/ allow it in the final position while only two-thirds of them allow it in the initial and medial position. 
Two most common ``rhotics'' in the data, /r/ and /ɾ/ also exhibit the Word-Initial Rhotic Avoidance to some degree, although it is more apparent for /ɾ/.

\par
The non-continuant segments (i.e. with the oral tract completely blocked) in this study can be divided into three main groups: nasal, voiced stop and voiceless stop. 
These groups are worth of consideration because all three major places of articulation (labial, coronal, and dorsal) of them appear in the top 20 most common segments of the study. 
In general, the nasal segments seem to be more symmetric than the voiceless, which in turn are more symmetric than the voiced.
The discrepancies lie mainly on the final position, since all of them (except /ŋ/) are nearly universal in initial and medial position.
The voiced stop is the most restricted group in this position, only allowed in under one third of the languages in which they occur. They are followed by the voiceless stops, which is allowed in the final position for around a half of the languages that they appear in. The least restricted are the nasals, which are near universal also in the final position.


\par
Turning to the fricatives, we find three fricative segments in the top 20: /s/, /f/ and /h/. 
There is a large difference in the frequency of /s/, which occurs in 18 languages compared to the other two fricatives, which occur in only 10 languages each. 
However, the behaviour of /s/ and /f/ are fairly similar to the voiceless stops. 
They are also near universal in the initial and medial position and are allowed in the final position in over a half of their occurrence. 
Though /s/ appear slightly more favoured in the final position than initial position. The glottal fricative /h/, on the other hand, are much more restricted in the final position. 

\par
The non-nasal sonorants in this data are comprised of /l j w r ɾ/. 
Most of them behaves similar to the voiceless stops, with near universal presence in the initial and medial, and allowed around half of the time in final position. 
The exception is the rhotic tap /ɾ/ which is much more restricted in the initial position while still universally accepted in the medial position. 
In fact initial is the most restricted position of /ɾ/, even more than the final position. 
The final position of /ɾ/, on the other hand, has the highest occurrence out of all non-nasal segments.

\par
The remaining miscellaneous segments are the aspirated stop /tʰ/, the glottal stop /ʔ/, and the affricate /tʃ/. 
The affricate is perhaps the most special segment out of the three. 
It is the only segment in the top-20 list that is totally inhibited from occurring in the final position. 
The glottal stop behaves similar to other (plain) voiceless stops, except that the occurrence in the initial position is slightly less than medial position. 
The voiceless aspirated /tʰ/, on the other hand, behaves more like voiced stops.
It has lower occurrence than all other voiceless stops in all position, including the final.

\section{Discussion}

The non-continuants /m n ŋ p t k b d g/, because of how frequent they are in languages of the world, are worth of a detailed discussion.
I will first discuss about the near universal acceptance of the nasal in the final position and try to speculate a possible explanation.
Then, I will look at the distribution of the voiceless and voiced stop.

\par
The near universal acceptance of the nasals in the final position can perhaps be attributed to them being sonorants.
In Optimality Theory (OT), there are two major markedness constraints regards syllable structure: \textsc{onset} and \textsc{nocoda}.
The first of them, \textsc{onset}, is about the initial position and thus not so relevant to our discussion about the distribution at the final position.
The second one, \textsc{nocoda}, could be the explanation for the phenomena we are observing.
The \textsc{nocoda} constraint states that a syllable should not end with a consonant.
Thus, the syllables that have coda will violate the constraint and become less optimal than syllables without it.
According to OT, only the optimal form, meaning form that violates the least constraints, will be chosen to produce by the speaker.
Only when the faithfulness constraint is ranked above the markedness constraint that a syllable with coda in the underlying form can have its coda produced.
Therefore, it can be said that the 'default' option for syllable structure in languages is without a coda.
\citet{Gordon_2016} also reported a similar observation, that if a language permits syllables with coda, it will also permit syllables without coda.

\par
But it is clear from the data that not all coda are treated equally. 
If a language allows its syllable to have coda, the three nasals are almost universally a candidate for that position, while other segments are generally much less favoured.
Perhaps not all segments violate the \textsc{nocoda} constraint to the same degree, and there is some attribute of the nasals that make them violate the constraint to a lesser extend.
The common feature that nasals are usually considered to share with vowels is that they are all sonorants.
We could speculate that languages prefer their syllables to have a more sonorant ending.
Vowels is clearly as sonorant as segments can get, but the other quite viable alternatives within this non-continuants group are the nasals.
So, the nasals violate the \textsc{nocoda} constraint less than stops and thus are more likely to be chosen as the candidate.

\par
The difference in the restriction of voiced and voiceless stops can be expected by the well-known Final Obstruent Devoicing process. 
Final Obstruent Devoicing means that voiced stops can be neutralized to their voiceless counterpart when they are at the end of the word. 
Another related phenomena is Intervocalic Voicing where voiceless stop becomes voiced when it is between vowels. 
However, the predicted effect of this phenomena, namely that voiceless stops are more restricted than voiced stops in medial position, is not observed in the top 20 segments.
The likelihood of voiced and voiceless segments in medial position is found to be fairly equal.
This is quite surprising because some accounts consider them to be similar phonological processes, namely weakening processes \citep{Gordon_2016, harris2009final}.
Both can be seen as reducing the articulation effort: it takes less effort to keep the adduction the vocal folds between two vowels and it takes more effort to maintain voicing at the end of the word \citep{Gordon_2016}. 
Thus, the account of a weakening process is not enough to explain the distribution of voiced and voiceless segments in this data.
It is possible that the efforts in the two cases are not the same and one has more priority than the other.
Indeed, voicing in vowels and consonants, although usually represented in phonology by the same feature, are different in terms of aerodynamic.
The voicing of vowels and sonorants is due to an absence of built-up air pressure in the oral tract while the voicing of obstruent requires such air pressure \citep{harris2009final}.
Thus, even when intervocalic voicing avoiding changing the adduction of the vocal folds (which is required for articulating vowels), it still has to increase the air pressure to make obstruent voicing possible.
So, intervocalic voicing can be seen as choosing between two efforts, changing the states of the vocal folds or increasing air pressure, while final devoicing only minimize efforts. 
That might be the reason why the preference for voicelessness of the stop in the final position is better reflected in the data.

\section{Interesting Findings}  

\subsection{The Glottal Stop}
The most phonetically favourable position for glottal stop is probably word-initial. 
Even languages like English, where glottal stop does not have phonemic status, still seem to favour word-initial glottal stop at the phonetic level \citep{garellekGlottalStop}. 
In fact, \citet{garellekGlottalStop} went as far as to suggest that perhaps all languages that does not contrast /\#V/ and /\#ʔV/ could insert [ʔ] at the beginning of the word, though they may do so with different degrees of frequency.

\par
However, the data gathered in my study shows that the glottal stop can be prohibited in the initial position. 
There is only one language that shows such behaviour, and that odd data point is Supyire, a language of the Niger-Congo family. 
This language's syllable structure is CV or CVV, with the exception of a few grammatical words that allow word-initial vowel \citep[7]{carlsonGrammarSupyire1994}. 
Given such syllable structure, one may expect that if the glottal stop is recognized in the language, its most favourable position would be word-initial.
However, the glottal stop is reported to be only a marginal phoneme in Supyire and can only appear in intervocalic environment. 
\citet{carlsonGrammarSupyire1994} speculated, based on evidence from loan words, that /ʔ/ (the author use the symbol /h/) might be a reduction of an earlier /g/.
But because in the current phonological system of the language there is also a /g/ that can appear in intervocalic position, /ʔ/ need to be considered a separate phoneme instead of an allophone of /g/.

\subsection{Final Avoidance of Two Nasals}

Although the nasals are supposed to be well-accepted in the final position, and the velar nasal is universally accepted, there are still two cases where /n/ or /m/ is not allowed in the coda. 
The first of them, the absence of /m/ in the coda, is observed in Mandarin Chinese.
The second one is the absence of final /n/, which is observed in Hausa, an Afro-Asiatic language.
The reason for the absence is different in each case: for Mandarin it is because of a historical merging process, while for Hausa it is allophony.

\par
The rhyme table for Mandarin Chinese presented in \citet{liMandarinChineseFunctional2009} contains only syllables that end with either a vowel or one of the two nasals /n/ and /ŋ/.
The labial nasal /m/, despite being a legit segment in the onset, does not appear in the coda.
More surprisingly, /m/ is a legit segment in the coda in Middle Chinese, and still remain so in a few other Sinitic languages like Cantonese \citep{zee1985sound}. 
The explanation given by \citet{zee1985sound} is that the labial nasal /m/ have merged into /n/ in Mandarin Chinese.
Middle Chinese allowed all three nasals /m/, /n/ and /ŋ/ to appear in the end of the word, but there is a merging process that affects all Sinitic languages.
In this process, the labial nasal /m/ seems to be the prime target for being merged, and in Mandarin it is merged with /n/ \citep{zee1985sound}.

\par
For the case of Hausa, it is more due oto methodological decision made in the process of data gathering. Specifically, \citet{newmanHausaLanguageEncyclopedic2000} presented Hausa as having only two phonemic nasal: /n/ and /m/. 
But he also noted that /n/ in the final position is always realized as [ŋ] and never as [n]. 
Therefore, in my data gathering Hausa is considered having both [n] and [ŋ], with [n] allowed in initial and medial and [ŋ] allowed in only final position.
The labial /m/ is also usually realized as [ŋ], but there are some cases where it is still realized as [m].
Thus, /m/ is still coded as being allowed in all three position in my data.

\par
Thus, the absence of word-final /m/ in Mandarin and /n/ in Hausa is not an evidence against the universal acceptance of nasals in word-final position.
They are some particular cases of historical changes or allophony within the space of the nasal inventory.
But another interesting question that they pose is "Why the velar nasal?".
Although in Mandarin /n/ still remain distinguished from /ŋ/, some other Sinitic languages also show a process of /n/ merging to /ŋ/ \citet{zee1985sound}.
Combined that with the distribution of /ŋ/ in word-final position, as observed in \citet{wals-9}, it seems like the velar nasal holds a strong preference in this position.
One possible speculation is that the velar nasal is the most vowel-like nasal.
And if it is true that languages prefer to end words with a vowel, then the velar nasal could be the most highly ranked alternative.
