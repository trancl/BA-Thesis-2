Despite the small language list used in this study, and thus the lack of significant statistical analysis, a few patterns can be observed from the data of the 20 most common segments.
We can first conclude that asymmetry is indeed the norm for most segments, with the exception of two nasals /n/ and /m/.
The nasals, if they exist in a language, are always allowed in the final position.
Two of them, /m/ and /n/, are also always allowed in the initial and medial position.
There are two cases where /m/ and /n/ are absent from the final position but they are due to a process of merging or allophony with another common nasals.
The velar nasal /ŋ/, on the other hand, is only universally allowed in the final position and is more restricted in other position.
While this observation conforms to the report from \cite{wals-9}, neither the fact that the distribution of /ŋ/ is asymmetric nor its degree of asymmetry is special.
However, what makes the distribution of /ŋ/ special is that it is the only segment that is skewed toward the final position.

\par
One particular question that remains is why the nasals are that well-accepted in the final position, and a possible reason could be due to its high sonority.
But that answer is still unsatisfying since other sonorants like liquids and glides are just as popular in the final position as the voiceless stops, which is on the other end of the sonority scale.
Perhaps the nasals are more similar to vowels than other sonorants in some attributes that are not considered in this thesis.

\par
Turning to the voiced and voiceless stops, the strong restriction on the voiced stops in the final position is well-expected by the Final Obstruent Avoidance process.
However, the effect of the related phenomena of intervocalic voicing is not observed in the data.
A speculated explanation is that maintaining voicing in word-final position takes more effort than keeping the stops voiceless between two vowels.
Therefore, the need to avoid voicing in word-final stops is more profound and reflected better in the data.

\par
Due to numerous limitation, many explanations of the data in this study have to remain at the speculation stage.
Nevertheless, it can serve as a preliminary research to look for direction for further studies.
For example, another research could try to compare the similarity between vowels and nasals versus vowels and other sonorants as a way to explain the high acceptance of nasals in word-final position.
